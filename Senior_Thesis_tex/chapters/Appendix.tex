\begin{appendices}

\section{Data Pre-processing}
\subsection{Stationarity}
\begin{minted}[
frame=lines,
framesep=2mm,
baselinestretch=1.2,
fontsize=\footnotesize,
linenos
]{python}
# Import modules
import pandas as pd
import numpy as np
from statsmodels.tsa.stattools import adfuller

# Load csv
df = pd.read_csv('df_master.csv')

# Define ADF test for multiple significance levels
def ad_test(df, col):
  name = str(col)
  dataset = df[col]
  result = True
  dftest = adfuller(dataset, autolag = 'AIC')
  sig_level = ['10%', '5%', '1%']
  for sig in sig_level:
    if dftest[0] > dftest[4][sig]:
      reject = sig
      result = False
      break
    else:
      pass
  if result == True:
    result_text = 'Pass'
  else:
    result_text = 'Reject at ' + sig
  return name + ' & ' + "{:.4f}".format(dftest[0]) + ' & ' + "{:.4f}".format(dftest[4]['1%']) 
        + ' & ' + "{:.4f}".format(dftest[4]['5%']) + ' & ' + "{:.4f}".format(dftest[4]['10%'])
        + ' & ' + result_text + "\\" +"\n"

# Run the function and store the results in a txt file
file = open("ADF_tabular.txt","w")
for col in df_norm.columns:
  file.writelines(ad_test(df_norm, col))
file.close()

# Take the First Difference for non-stationary processes
## Take the features of non-stationary process
non_stationary_feat = ['hospitalized', 'light-mid_symptoms', 'severe_symptoms', 'dead',
                        'PCR_negative', 'tested_MA(7days)']
df_copy = df[non_stationary_feat]

## Take the first difference
df_diff = df_copy.diff()
df_diff = df_diff.dropna()

## Redo the ADF test and store the results to a separate file
file2 = open("ADF_tabular2.txt","w")
for col in df_diff.columns:
  file2.writelines(ad_test(df_diff, col))
file2.close()

# Make a single dataframe with all stationary features
stat_cols = []
for col in df.columns:
  if col not in non_stationary_feat:
    stat_cols.append(col)
df_stat = df[stat_cols]
newdata = pd.merge(df_stat, df_diff, on = 'date')
newdata = newdata.reindex(columns = df.columns)
\end{minted}

\subsection{Multicollinearity}
\begin{minted}[
frame=lines,
framesep=2mm,
baselinestretch=1.2,
fontsize=\footnotesize,
linenos
]{python}
# Import modules
from statsmodels.stats.outliers_influence import variance_inflation_factor as VIF

# Prepare dataframe 
newdf = newdata.copy()
## Drop the objective function
newdf = newdata.drop(columns = {'positive_rate'})

# Define the VIF generator from a given dataframe 
def VIF_operator(df):
  X_variables = df.copy()
  vif_data = pd.DataFrame()
  vif_data["feature"] = df.columns
  vif_data["VIF"] = [VIF(X_variables.values, i) for i in range(len(X_variables.columns))]
  return vif_data

# Run the function for dataframe
vif1 = VIF_operator(newdf)

# Store the results in txt file
file2 = open('VIF_result1.txt', 'w')
for i in range(11):
  text = str(vif1.values[i][0]) + ' & ' + str(vif1.values[i][1]) + " \\" + "\\" + '\n'
  file2.write(text)
file2.close()

# From the results drop identical features
newdata1 = newdf.copy()
newdata1 = newdata1.drop(columns = ['hospitalized', 'comparisonPreDeclare'])
newdata1.index = pd.to_datetime(newdata1.index)

# Recalculate VIF
vif2 = VIF_operator(newdata1)

#Store the results in txt file
file3 = open('VIF_result2.txt', 'w')
for i in range(9):
  text = str(vif2.values[i][0]) + ' & ' + str(vif2.values[i][1]) + " \\" + "\\" + '\n'
  file3.write(text)
file3.close()

# Store the Pre-processed data set to a csv file
## Get the dataframe which includes the objective function
positive = df.loc[df['positive_rate', 'comparisonPreDeclare']]
df_master_final = pd.merge(positive, newdata1, on = 'date')
## Drop unnecessary column
df_master_final = df_master_final.drop(columns={'comparisonPreDeclare'})
## store the result to "df_master_final.csv"
df_master_final.to_csv('df_master_final.csv')
\end{minted}

\section{Conventional Model}
\subsection{ARMA model}
\begin{minted}[
frame=lines,
framesep=2mm,
baselinestretch=1.2,
fontsize=\footnotesize,
linenos
]{python}
# Import necessary modules
import pandas as pd
import numpy as np
import matplotlib.pyplot as plt
from pmdarima import auto_arima
import warnings
from statsmodels.tsa.arima.model import ARIMA
from copy import deepcopy as dc
from sklearn.metrics import mean_squared_error
from math import sqrt
plt.rcParams["font.family"] = "serif"
% matplotlib inline
warnings.filterwarnings('ignore')

# Read data
df = pd.read_csv('df_master_final.csv', index_col='date')
df.index = pd.to_datetime(df.index)

# Calculate the lags using 'auto_arima' module
## Optimal lags will be calculated which minimizes the AIC
stepwise_fit = auto_arima(df['Positive\_Rate'], trace=True, suppress_warnings = True)
stepwise_fit.summary()

# 40 days prediction
## Split dataset
train40 = df.iloc[:-40]
test40 = df.iloc[-40:]
model = ARIMA(train40['Positive\_Rate'], order = (2, 1, 2))
model = model.fit()

## Make a prediction and store to a dataframe
start = len(train40)
end  = len(df)
pred = model.predict(start=start, end=end-1, typ='levels')
pred.index = df.index[start:end+1]
df_pred = pd.DataFrame(pred)
df_pred = df_pred.rename(columns={'predicted_mean': 'Positive\_Rate'})

# Make a line graph of the prediction
## Prepare the dataframes
df_plot = df[df.columns[0]]
df_pred_prev = pd.DataFrame(df_plot[-100:-40])
df_plot = pd.DataFrame(df_plot[-100:])
df_pred_plot = pd.concat([df_pred_prev, df_pred])

# Caclate the RMSE
rmse = sqrt(mean_squared_error(df_pred, test40['Positive\_Rate']))

## Generate graph
ax = df_plot.plot(figsize=(10,8), legend=True)
df_pred_plot.plot(ax=ax, figsize=(10,8), legend=True)
plt.xlabel('Timestamps')
plt.ylabel('Positive Rate')
plt.legend(['Predicted','Observed'])
### save figure
plt.savefig('ARIMA(2,1,2)_40Days.png', dpi=100)
\end{minted}


\subsection{VARMA model}
\begin{minted}[
frame=lines,
framesep=2mm,
baselinestretch=1.2,
fontsize=\footnotesize,
linenos
]{python}
# Import modules
from statsmodels.tsa.statespace.varmax import VARMAX
from pmdarima import auto_arima
from sklearn import metrics
from timeit import default_timer as timer
import warnings
warnings.filterwarnings("ignore")

# Prepare the used dataframe
df1 = pd.read_csv('df_master_final.csv', index_col='date')
df1.index = pd.to_datetime(df1.index)
data = df1.copy()
## Drop some columns to truncate the calculation time necessary for VARMA modelling
data = data.drop(columns={'Dead', 'PCR_negative', 'ComparisonPreDay', 'Tested\_MA(7days)',
                          'Light-Mid\_Symptoms', 'Severe\_Symptoms'})]

# Define a function which shows the evaluation metric
def timeseries_evaluation_metrics_func(y_true, y_pred):    
    def mean_absolute_percentage_error(y_true, y_pred): 
        y_true, y_pred = np.array(y_true), np.array(y_pred)
        return np.mean(np.abs((y_true - y_pred) / y_true)) * 100
    print('Evaluation metric results:-')
    print(f'MSE is : {metrics.mean_squared_error(y_true, y_pred)}')
    print(f'MAE is : {metrics.mean_absolute_error(y_true, y_pred)}')
    print(f'RMSE is : {np.sqrt(metrics.mean_squared_error(y_true, y_pred))}')
    print(f'MAPE is : {mean_absolute_percentage_error(y_true, y_pred)}')
    print(f'R2 is : {metrics.r2_score(y_true, y_pred)}',end='\n\n')

# Define a function which takes the inverse of a given column in a dataframe
def inverse_diff(actual_df, pred_df):
    df_res = pred_df.copy()
    columns = actual_df.columns
    for col in columns: 
        df_res[str(col)+'_1st_inv_diff'] = actual_df[col].iloc[-1] + df_res[str(col)].cumsum()
    return df_res

# 40Days Prediction
X = data.copy()
train, test = X[0:-40], X[-40:]
cols = list(train.columns)

## Get the lag length by calculating the AIC for all the used features
pq = []
for name, column in train[['Positive\_Rate', 'Discharged', 
                           'PCR\_Positive','ComparisonPreSpread']].iteritems():
    print(f'Searching order of p and q for : {name}')
    stepwise_model = auto_arima(train[name],start_p=1, start_q=1,max_p=10, max_q=10, seasonal=False,
        trace=True,error_action='ignore',suppress_warnings=True, stepwise=True,maxiter=1000)
    parameter = stepwise_model.get_params().get('order')
    print(f'optimal order for:{name} is: {parameter} \n\n')
    pq.append(stepwise_model.get_params().get('order'))

## VARMA modelling for the calculated lags and show the RMSE for the predicted results
df_results_moni = pd.DataFrame()
for i in pq:
    if i[0]== 0 and i[2] ==0:
        pass
    else:
        print(f' Running for {i}')
        model = VARMAX(train[['Positive\_Rate', 'Discharged', 'PCR\_Positive', 
                              'ComparisonPreSpread']], order=(i[0],i[2])).fit(disp=False)
        result = model.forecast(steps = 40)
        inv_res = inverse_diff(data[['Positive\_Rate', 'Discharged', 'PCR\_Positive',
                                     'ComparisonPreSpread']] , result)
        rmse = np.sqrt(metrics.mean_squared_error(test['Positive\_Rate'], inv_res['Positive\_Rate']))
        df_results_moni = df_results_moni.append({'p': i[0], 'q': i[2],
                                                  'RMSE Positive Rate':rmse}, ignore_index=True)

## Sort the models in descending order according to the RMSE
df_results_moni.sort_values(by = ['RMSE Positive Rate'])

## We use the VARMA model which gives the least RMSE value
model = VARMAX(train[['Positive\_Rate', 'Discharged', 'PCR\_Positive',
                      'ComparisonPreSpread']], order=(2,3)).fit( disp=False)
result = model.forecast(steps = 40)
res = inverse_diff(data[['Positive\_Rate', 'Discharged', 'PCR\_Positive','ComparisonPreSpread']],result)
df_pred = pd.DataFrame(res['Positive\_Rate'])
original = pd.DataFrame(df1['Positive\_Rate'][-100:])
## Get the RMSE of the predicted values
rmse = sqrt(mean_squared_error(df_pred, original[-40:]))
## Prepare the dataframe to plot
original_pred = pd.DataFrame(df1['Positive\_Rate'][-100:-40])
df_pred_plot = pd.concat([original_pred, df_pred])

## plot the prediction data
ax = original.plot(legend=True)
df_pred_plot.plot(ax = ax, legend=True)
plt.legend(['Real Data', 'Predicted Data'])
plt.xlabel('Timestamp')
plt.ylabel('Positive Rate')
### save the figure
plt.savefig('VARMA(2,3)_40Days.png', dpi=100)
\end{minted}

\section{Neural Networks}
\subsection{Univariate Prediction}
\begin{minted}[
frame=lines,
framesep=2mm,
baselinestretch=1.2,
fontsize=\footnotesize,
linenos
]{python}
# Importing the libraries
import pandas as pd
import numpy as np
import matplotlib.pyplot as plt
from sklearn.preprocessing import MinMaxScaler
from keras.models import Sequential
from keras.layers import Dense, LSTM, Dropout, GRU, LeakyReLU
from tensorflow.keras.optimizers import SGD
import math
from sklearn.metrics import mean_squared_error

# Preparing data
df = pd.read_csv('df_master_final.csv', index_col='date')
df.index = pd.to_datetime(df.index)

# Some functions to help out with
def plot_predictions(test, predicted, name):
    figure_name = str(name)+'.png'
    plt.plot(test,label='Real Positive Rate')
    plt.plot(predicted,label='Predicted Positive Rate')
    plt.xlabel('Timestamp')
    plt.ylabel('Positive Rate')
    plt.legend()
    plt.savefig(figure_name, dpi=100)

def return_rmse(test,predicted):
    rmse = math.sqrt(mean_squared_error(test, predicted))
    print("The root mean squared error is {}.".format(rmse))

# LSTM 40 Days

## Preparing data
### Splitting data
training_set = np.array(df['Positive\_Rate'][:-40])
training_set = training_set.reshape(-1, 1)
test_set = np.array(df['Positive\_Rate'][-40:])
test_set = test_set.reshape(-1, 1)
limit = len(training_set)
### Scaling the training set
sc = MinMaxScaler(feature_range=(0,1))
training_set_scaled = sc.fit_transform(training_set)
### We create a data structure with 15 timesteps and 1 output
X_train = []
y_train = []
memory_days = 15
for i in range(memory_days, limit):
    X_train.append(training_set_scaled[i-10:i,0])
    y_train.append(training_set_scaled[i,0])
X_train, y_train = np.array(X_train), np.array(y_train)
### Reshaping X_train for efficient modelling
X_train = np.reshape(X_train, (X_train.shape[0],X_train.shape[1],1))

## The LSTM architecture
regressor40 = Sequential()
### First LSTM layer with Dropout regularisation
regressor40.add(LSTM(units=128, return_sequences=True, input_shape=(X_train.shape[1],1)))
regressor40.add(LeakyReLU(alpha=0.181818))
regressor40.add(Dropout(0.4))
### Second LSTM layer
regressor40.add(LSTM(units=256, return_sequences=True))
regressor40.add(LeakyReLU(alpha=0.181818))
regressor40.add(Dropout(0.4))
### Third LSTM layer
regressor40.add(LSTM(units=128, return_sequences=True))
regressor40.add(LeakyReLU(alpha=0.181818))
regressor40.add(Dropout(0.4))
### Fourth LSTM layer
regressor40.add(LSTM(units=64))
regressor40.add(LeakyReLU(alpha=0.181818))
regressor40.add(Dropout(0.4))
### The output layer
regressor40.add(Dense(units=1))
### Compiling the RNN
regressor40.compile(optimizer='adam',loss='mean_squared_error')
### Fitting to the training set
history40 = regressor40.fit(X_train,y_train,epochs=500,batch_size=32, validation_split=0.1)

## Plot the accuracy preformance results
plt.plot(history40.history['loss'], label='Training loss')
plt.plot(history40.history['val_loss'], label='Validation loss')
plt.xlabel('Epochs')
plt.ylabel('MSE')
plt.legend()
plt.savefig('LSTM_Uni_40Days_Training.png', dpi=100)

## Prepare prediction
dataset_total = pd.concat((df[:-40]['Positive\_Rate'],df[-40:]['Positive\_Rate']),axis=0)
inputs = dataset_total[len(dataset_total)- len(test_set) - memory_days:].values
inputs = inputs.reshape(-1,1)
inputs = sc.transform(inputs)
X_test = []
for i in range(memory_days,40+memory_days):
    X_test.append(inputs[i-10:i,0])
X_test = np.array(X_test)
X_test = np.reshape(X_test, (X_test.shape[0],X_test.shape[1], 1))

## Store the prediction results to a dataframe
predicted_positive_rate = regressor40.predict(X_test)
predicted_positive_rate = sc.inverse_transform(predicted_positive_rate)
pred = predicted_positive_rate.copy()
pred.reshape(40)
date = pd.date_range('2021-11-21', '2021-12-30')
df_pred = pd.DataFrame(pred)
df_pred.index = date
df_pred.index.name = 'date'
df_pred = df_pred.rename(columns={'[0]':'Predicted_Positive\_Rate'})

## Evaluating our LSTM model
test_test = pd.DataFrame(df['Positive\_Rate'][-100:])
train_test = df['Positive\_Rate'][:-40]
pred_test1 = pd.concat([train_test, df_pred])
pred_test = pred_test1[-100:]
test_array = np.array(test_test)
pred_array = np.array(pred_test)
return_rmse(test_array, pred_array)

## Visualizing the results for LSTM
plot_predictions(test_test, pred_test, 'LSTM_Uni_40Days')

# GRU 40 Days
## The GRU architecture
regressorG40 = Sequential()

### First GRU layer with Dropout regularisation
regressorG40.add(GRU(units=128, return_sequences=True, input_shape=(X_train.shape[1],1)))
regressorG40.add(LeakyReLU(alpha=0.181818))
regressorG40.add(Dropout(0.4))

### Second GRU layer
regressorG40.add(GRU(units=256, return_sequences=True))
regressorG40.add(LeakyReLU(alpha=0.181818))
regressorG40.add(Dropout(0.4))

### Third GRU layer
regressorG40.add(GRU(units=128, return_sequences=True))
regressorG40.add(LeakyReLU(alpha=0.181818))
regressorG40.add(Dropout(0.4))

### Fourth GRU layer
regressorG40.add(GRU(units=64))
regressorG40.add(LeakyReLU(alpha=0.181818))
regressorG40.add(Dropout(0.4))

### The output layer
regressorG40.add(Dense(units=1))

### Compiling the RNN
regressorG40.compile(optimizer='adam',loss='mean_squared_error')

### Fitting to the training set
historyG40 = regressorG40.fit(X_train,y_train,epochs=500,batch_size=32, validation_split=0.1)

## Plot the accuracy preformance results
plt.plot(historyG40.history['loss'], label='Training loss')
plt.plot(historyG40.history['val_loss'], label='Validation loss')
plt.xlabel('Epochs')
plt.ylabel('MSE')
plt.legend()
plt.savefig('GRU_Uni_40Days_Training.png', dpi=100)

## Predicting values
predicted_positive_rateG = regressorG40.predict(X_test)
predicted_positive_rateG = sc.inverse_transform(predicted_positive_rateG)
predG = predicted_positive_rateG.copy()
predG.reshape(40)
predG = pd.date_range('2021-11-21', '2021-12-30')
df_predG = pd.DataFrame(predG)
df_predG.index = date
df_predG.index.name = 'date'
df_predG = df_predG.rename(columns={'[0]':'Predicted_Positive\_Rate'})

## Evaluating our GRU model
test_test = pd.DataFrame(df['Positive\_Rate'][-100:])
train_test = df['Positive\_Rate'][:-40]
pred_test1 = pd.concat([train_test, df_predG])
pred_test = pred_test1[-100:]
test_array = np.array(test_test)
pred_array = np.array(pred_test)
return_rmse(test_array, pred_array)

## Visualizing the results for GRU
plot_predictions(test_test, pred_test, 'GRU_Uni_40Days')
\end{minted}

\subsection{Multivariate Prediction}
\begin{minted}[
frame=lines,
framesep=2mm,
baselinestretch=1.2,
fontsize=\footnotesize,
linenos
]{python}
# Import modules
import pandas as pd
import numpy as np
from sklearn.metrics import mean_squared_error
from math import sqrt
import matplotlib.pyplot as plt
from matplotlib.pyplot import figure
from tensorflow.keras.models import Sequential
from tensorflow.keras.layers import LSTM, GRU, Dense, Dropout, LeakyReLU
from sklearn.preprocessing import StandardScaler
import seaborn as sns
plt.rcParams["font.family"] = "serif"
% matplotlib inline

# Prepare dataframes
df1 = pd.read_csv(path +'df_master_final.csv', index_col='date')
df1.index = pd.to_datetime(df1.index)
df = df1.copy()
## Drop if modification are made
#df = df.drop(columns={'Dead', 'PCR_negative', 'ComparisonPreDay', 'Tested\_MA(7days)', 
                      #'Light-Mid\_Symptoms', 'ComparisonPreSpread', 'Severe\_Symptoms'})

# 40 Days Prediction LSTM (Multivariate)
## Split data
df_training = df[:-40].copy()
df_master = df.copy()

## Separate dates for future plotting
train_dates = pd.DataFrame(df_training.index)
predict_period_dates = pd.DataFrame(df_master[-40:].index)

## Variables for training
cols = list(df_master)[0:10]

## Plot the used data features
df_for_training = df_training[cols].astype(float)
df_for_plot=df_for_training.tail(100)
df_for_plot.plot.line()

## normalize the dataset
scaler = StandardScaler()
scaler = scaler.fit(df_for_training)
df_for_training_scaled = scaler.transform(df_for_training)

## Prepare training data
trainX = []
trainY = []
### Number of days we look into the future and the past days we use to predict the future
n_future = 1
n_past = 14 
### Reformat input data into a shape
for i in range(n_past, len(df_for_training_scaled) - n_future +1):
    trainX.append(df_for_training_scaled[i - n_past:i, 0:df_for_training.shape[1]])
    trainY.append(df_for_training_scaled[i + n_future - 1:i + n_future, 0])
trainX, trainY = np.array(trainX), np.array(trainY)

## define the LSTM model
model40 = Sequential()

### First Layer
model40.add(LSTM(128, input_shape=(trainX.shape[1], trainX.shape[2]), return_sequences=True))
model40.add(LeakyReLU(alpha=0.181818))
model40.add(Dropout(0.4))

### Second Layer
model40.add(LSTM(512, return_sequences=True))
model40.add(LeakyReLU(alpha=0.181818))
model40.add(Dropout(0.4))

### Third Layer
model40.add(LSTM(256, return_sequences=True))
model40.add(LeakyReLU(alpha=0.181818))
model40.add(Dropout(0.4))

### Fourth Layer
model40.add(LSTM(64, return_sequences=False))
model40.add(LeakyReLU(alpha=0.181818))
model40.add(Dropout(0.4))

### Output Layer
model40.add(Dense(trainY.shape[1]))

### Compile
model40.compile(optimizer='adam', loss='mse')

### Fit the model
history40 = model40.fit(trainX, trainY, epochs=500, batch_size=32, validation_split=0.1, verbose=1)

## Plot the accuracy preformance results
plt.plot(history40.history['loss'], label='Training loss')
plt.plot(history40.history['val_loss'], label='Validation loss')
plt.xlabel('Epochs')
plt.ylabel('MSE')
plt.legend()
plt.savefig('LSTM_Mult_40Days_Training.png', dpi=100)

## Make prediction
n_past = 40
n_days_for_prediction=40  
prediction = model40.predict(trainX[-n_days_for_prediction:])
prediction_copies = np.repeat(prediction, df_for_training.shape[1], axis=-1)
y_pred_future = scaler.inverse_transform(prediction_copies)[:,0]
### Convert the data into a dataframe
df_forecast = pd.DataFrame({'date':predict_period_dates.values.T[0], 'Positive\_Rate':y_pred_future})
df_forecast['date']=pd.to_datetime(df_forecast['date'])
df_forecast.index = df_forecast['date']
df_forecast = df_forecast.drop(columns={'date'})
original = pd.DataFrame(df_master['Positive\_Rate'][-100:])
original_pred = pd.DataFrame(df_master['Positive\_Rate'][-100:-40])
df_pred_plot = pd.concat([original_pred, df_forecast])
### Evaluate our LSTM model
rmse = sqrt(mean_squared_error(df_forecast, original[-40:]))

## Plot the prediction result
sns.lineplot(data=original, x=original.index, y=original['Positive\_Rate'])
sns.lineplot(data=df_pred_plot, x=df_pred_plot.index, y=df_pred_plot['Positive\_Rate'])
plt.xlabel('Timestamp')
plt.ylabel('Positive Rate')
plt.legend(['Real Data', 'Predicted Data'])
### Save figure
plt.savefig('LSTM_Mult_40Days.png', dpi=100)

# 40 Days Prediction GRU (Multivariate)
## define the GRU model
modelG40 = Sequential()

### First Layer
modelG40.add(GRU(128, input_shape=(trainX.shape[1], trainX.shape[2]), return_sequences=True))
modelG40.add(LeakyReLU(alpha=0.181818))
modelG40.add(Dropout(0.4))

### Second Layer
modelG40.add(GRU(512, return_sequences=True))
modelG40.add(LeakyReLU(alpha=0.181818))
modelG40.add(Dropout(0.4))

### Third Layer
modelG40.add(GRU(256, return_sequences=True))
modelG40.add(LeakyReLU(alpha=0.181818))
modelG40.add(Dropout(0.4))

### Fourth Layer
modelG40.add(GRU(64, return_sequences=False))
modelG40.add(LeakyReLU(alpha=0.181818))
modelG40.add(Dropout(0.4))

### Output Layer
modelG40.add(Dense(trainY.shape[1]))

### Compile
modelG40.compile(optimizer='adam', loss='mse')

### Fit the model
historyG40 = modelG40.fit(trainX, trainY, epochs=500, batch_size=32, validation_split=0.1, verbose=1)

## Plot the accuracy preformance results
plt.plot(historyG40.history['loss'], label='Training loss')
plt.plot(historyG40.history['val_loss'], label='Validation loss')
plt.xlabel('Epochs')
plt.ylabel('MSE')
plt.legend()
plt.savefig('GRU_Mult_40Days_Training.png', dpi=100)

## Make prediction
n_past = 40
n_days_for_prediction=40  
predictionG = modelG40.predict(trainX[-n_days_for_prediction:])
prediction_copies = np.repeat(predictionG, df_for_training.shape[1], axis=-1)
y_pred_future = scaler.inverse_transform(prediction_copies)[:,0]
### Convert the data into a dataframe
df_forecast = pd.DataFrame({'date':predict_period_dates.values.T[0], 'Positive\_Rate':y_pred_future})
df_forecast['date']=pd.to_datetime(df_forecast['date'])
df_forecast.index = df_forecast['date']
df_forecast = df_forecast.drop(columns={'date'})
original = pd.DataFrame(df_master['Positive\_Rate'][-100:])
original_pred = pd.DataFrame(df_master['Positive\_Rate'][-100:-40])
df_pred_plot = pd.concat([original_pred, df_forecast])
### Evaluate our LSTM model
rmse = sqrt(mean_squared_error(df_forecast, original[-40:]))

## Plot the prediction result
sns.lineplot(data=original, x=original.index, y=original['Positive\_Rate'])
sns.lineplot(data=df_pred_plot, x=df_pred_plot.index, y=df_pred_plot['Positive\_Rate'])
plt.xlabel('Timestamp')
plt.ylabel('Positive Rate')
plt.legend(['Real Data', 'Predicted Data'])
### Save figure
plt.savefig('GRU_Mult_40Days.png', dpi=100)
\end{minted}
\end{appendices}