\section{Introduction}

The spread of the novel COVID-19 virus has caused wold-wide turmoil and has been a pressing issue since 2020 when World Health Organization (WHO) declared pandemic. Government officials has been forced upon taking immediate action to contain the highly infectious virus and there is high demand where we project statistics which reflects what the situation will be like in the near future. In reality, the Kanagawa prefecture officials and the Kanagawa University of Human Services has established a project team and released several predictive models which projects the number of people who have severe symptoms \citep{Kanagawa_prediction}. This model has been released in the Kanagawa prefecture's website and is intended for local citizens to have general understanding of what the situation will be like. Thus, having an accurate predicted model is highly desired for both the government officials; having more resources to base their political measures and for the citizens; acquiring prior knowledge of what the situation would be like in the near future. The motivation behind this paper is to reproduce a likewise model for Tokyo prefecture, which predicts the number of COVID-19 positive rates. 

Predicting future values using time series modelling is an intensively researched topic amongst academia and business. The most well known predictive model is the Autoregressive Integrated Moving Average (ARIMA) which has been popularly used due to its flexibility in its statistical properties. However, the obstacle in which ARIMA model presents to capture the nonlinear characteristics in data, have characterized the need of an alternative predictive model which captures the volatile nature of time series data \citep{Time_Series_forecasting_ARIMA}. With the rise of Artificial Neural Networks (ANN), the Long Short-Term Memory (LSTM) model have been presenting strong accuracy in its predictive abilities. Firstly introduced by \cite{LSTM} in 1997 as a special case of Recurrent Neural Network(RNN), many researches are comparing the accuracy of predictions with ARIMA models and LSTM models. Some research concluded that LSTM outperformed the conventional ARIMA model of predicting the stock prices by a high margin of 84-87\% reduction in error rates \citep{Comparison_ARIMA_LSTM}. Another breakthrough in predictive models occurred in 2014, when \citet{GRU} introduced the Gated Recurrent Unit (GRU) initially to capture the dependencies within sequences in the context of machine translation. Other researches have shown that GRU can be applied to time series data to forecast future values and have concluded GRU to be a superior predictive model over other ANN models \citep{GRU_performance}. 

This paper aims to utilize several deep learning models as well as conventional time series forecasting models in search for an accurate predictive model for Tokyo's COVID-19 positive rate. Using multiple features which has some correspondence with the value of COVID-19 positive rates, this paper will split the prediction question to two sections: Univariate modelling which only uses the objective value for training and testing, and Multivariate modelling which takes multiple time series features into account for training to produce the predictions. The core deep learning model that will be used in this paper are simple Recurrent Neural Network(RNN), Long Short-term Memory (LSTM), and Gated Recurrent Unit (GRU). For the baseline models, this paper will use a robust linear time series prediction models, which are the Autoregressive Integrated Moving Average (ARIMA) model for univariate modelling and Vector Autoregression (VAR) model for multivariate modelling. Utilizing these models, we compare the results of predictions to suggest the optimal predictive model. 

The specific implementation of these models as well as the pre-processing of the time series dataset will be covered in the \textit{Methodology} section, and the performance will be evaluated in the \textit{Results} section. The discussion and topics which needs to be coverred in the future will be mentioned in the \textit{Discussion and Future Works} section. 