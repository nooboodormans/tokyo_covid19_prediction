\thispagestyle{plain}
The novel COVID-19 has presented unprecedented impact over the world. Tokyo Japan is no different and the deadly virus has forced administrative management to take immediate measures to contain the virus. Therefore, prediction of COVID-19 patients and the situation in large has been a pressing research interest over the world. The feature in which the number of reported COVID-19 cases have some dependency with earlier observed statistical data, this paper aims to use multiple time series analysis approach to construct an effective predictive model of Tokyo's observed positive rate from PCR testing. This paper will focus on two predictive methods which are Univariate prediction and Multivariate prediction using several conventional models and neural network models. The models used in this paper are: Autoregressive Moving Average model (ARMA), Vector Autoregression Moving Average model(VARMA), Long Short-term Memory(LSTM) model, and Gated Recurrent Unit (GRU) model. These models are known to be robust when constructing predictive models. Using these models and predictive procedures, this paper will predict the observed positive rates of COVID-19 for the latest 40 days for standard prediction, 20 days for short-term prediction and 80 days for long-term prediction. This paper will also discuss the statistical assumptions it needs to pass for a clear prediction which are: stationarity within the input data, and checking for multicollinearity.

The result show that while linear models present some promising predictions, some neural network models were able to predict the model with a very high accuracy rate. This paper also found that the long term prediction of COVID-19 using neural network was difficult. Another difficulty was found during the comparison of Multivariate predictions and Univariate predictions, where Multivariate prediction failed to perform at the standard of accuracy which Univariate prediction was capable of. Extensive research in this paper have found that the foot traffic data, which accounts for people's movement, do not demonstrate a significant factor in the prediction process. 
